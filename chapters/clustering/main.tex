\documentclass[a4paper,12pt]{article}
\usepackage[T2A]{fontenc}
\usepackage[utf8]{inputenc}
\usepackage[english,russian]{babel}
\usepackage{amsmath,amsfonts,amssymb,amsthm,mathrsfs,mathtools}
\usepackage[colorlinks, linkcolor = purple, citecolor = purple]{hyperref}
\usepackage{xcolor}
\usepackage{xpatch}
\usepackage{marvosym}
\usepackage{cancel}
\usepackage{floatrow}
\usepackage{commath}
\usepackage{upgreek}
\usepackage{lipsum}
\usepackage{mhchem}
\usepackage{chemfig}
\usepackage{multirow}
\usepackage{tabularx}
\usepackage{tikz}
\usepackage{titletoc}
\usepackage{pgfplots}
\usepackage{wrapfig}
\usepackage{chngcntr}
\usepackage{makecell}
\usepackage{stackengine,graphicx}
\usepackage{cmap}
\usepackage{indentfirst}
\usepackage{tocloft}
\usepackage{setspace}
\usepackage{titlesec}
\usepackage{soul}
\usepackage[stable]{footmisc}
\usepackage{tocloft}
\usetikzlibrary{positioning}
\usepackage{caption}
\usepackage{subfig}
\pgfplotsset{width=10cm,compat=1.9}
\tikzset{>=stealth}
\usepackage[left=2cm,right=2cm,top=2cm,bottom=3cm,bindingoffset=0cm]{geometry}
%\DeclareMathOperator*{\esssup}{ess\,sup}
\DeclareMathOperator*{\tr}{tr}
\DeclareMathOperator*{\Ker}{Ker}
\DeclareMathOperator*{\Rea}{Re}
\DeclareMathOperator*{\Ima}{Im}
\DeclareFontEncoding{LS2}{}{\noaccents@}
\DeclareFontSubstitution{LS2}{stix}{m}{n}
\DeclareSymbolFont{arrows3}{LS2}{stixtt}{m}{n}
\DeclareMathSymbol{\squareulblack}{\mathord}{arrows3}{"88}
\date{\vspace{-10pt}}
\author{Дорогинин Д.В. Б02-825бф\\
Матвеев Г.А. Б02-824бф}
\title{\textbf{Электронный парамагнитный резонанс.}}


\theoremstyle{definition}
\newtheorem*{definition}{Определение}
\newtheorem{statement}{Предложение}
\newtheorem{lemma}{Лемма}
\newtheorem{theorem}{Теорема}
\newtheorem*{theorem*}{Теорема}
\newtheorem*{corollary}{Следствие}
\newtheorem*{example}{Пример}
\setcounter{tocdepth}{2}
%\renewcommand\cftsecafterpnum{\vspace{-pt}}
%\renewcommand\cftsecafterpnum{\vspace{1pt}}
\newcommand*{\eqdef}{\mathop{\overset{\mathrm{def}}{\resizebox{\widthof{\ensuremath{\mathop{\overset{\mathrm{def}}{=}}}}}{\heightof{=}}{=}}}}
\renewcommand\qedsymbol{$\squareulblack$}
\renewcommand{\cftsecleader}{\cftdotfill{\cftdotsep}}
\newfloatcommand{capbtabbox}{table}[][\FBwidth]
\newcommand{\HH}{\mathcal{H}}
\newcommand{\DD}{\mathcal{D}}
\newcommand{\LL}{\mathcal{L}}
\newcommand{\AAA}{\mathscr{A}}
\newcommand{\SpA}{\mathcal{A}}
\newcommand{\EE}{\mathcal{E}}
\newcommand{\suml}{\sum\limits_{n=1}^\infty}
\newcommand{\sumlN}{\sum\limits_{n=1}^N}
\newcommand{\sumo}{\sum\limits_{n=0}^\infty}
\newcommand{\sumoN}{\sum\limits_{n=0}^N}
\renewcommand\cftsecfont{\normalsize}
\renewcommand\cftsubsecfont{\normalsize}
\titleformat{\section}
 {\normalfont\fontsize{16}{16}\bfseries}{\thesection}{1em}{}
\def\rlwd{.5pt}
\def\crossy{\kern.5pt\def\stacktype{L}%
\stackon[0.65ex]{%
 \stackon[1.4ex]{%
   \stackon[1.1ex]{\rule{\rlwd}{1.8ex}}{\rule{1.4ex}{\rlwd}}%
 }{\rule{0.8ex}{\rlwd}}%
}{\rotatebox{-20}{\rule{0.8ex}{\rlwd}}}%
\kern1pt}



\newcommand{\angstrom}{\mbox{\normalfont\AA}}


\begin{document}
\stepcounter{page}
\section{DBSCAN}

\textbf{DBSCAN} (Density-Based Spatial Clustering of Applications with Noise) - алгоритм кластеризации, решающий проблему сО сферичностью кластеров, он не делает никаких предположений о форме кластеров. Также он довольно быстрый и подходит для кластеризации больших данных.
\\
Он основан на понятии {\textit{окрестности}}.

\textbf{Определение 1.} Задан объект $x \in U$, его $\varepsilon$-окрестность $U_\varepsilon (x) = \{\;u\in U:\; \rho (x,u) \leq \varepsilon \;\}$ - это множество объектов, которые находятся на расстоянии не больше $\varepsilon$ от заданного объекта $x$.

Тогда каждый объект может быть отнесен к одному из трёх типов:
\begin{itemize}
    \item \textit{корневой}: имеющий плотную окрестность,  {$\abs{U_\varepsilon (x)} \geq m$}, т.е. $\varepsilon$ содержит $\geq m$ объектов.
    \item \textit{граничный}: не корневой, но в окрестности корневого.
    \item \textit{шумовой (выброс)}: не корневой и не граничный.
\end{itemize}
\begin{figure}[h!]
    \centering
    \includegraphics[width=0.9\linewidth]{png/An-Example-Illustrating-the-Density-Based-DBSCAN-Clustering-Method-Applied-to-SMLM-Data.png}
    \caption{An Example Illustrating the Density-Based DBSCAN Clustering Method Applied to SMLM Data}
    \label{fig:enter-label}
\end{figure}
Возникает 2 параметра: $\varepsilon$ и $m$. Других параметров не будет. От этих параметров и будет зависеть то, какой картина кластеризации получится. Также к преимуществам этого метода относится то, что он не задает заранее количество кластеров, в отличие, например, от k-means, причём количество кластеров будет зависеть от $\varepsilon$ и $m$. 

Как работает алгоритм: берётся произвольная точка, если она имеет плотную окрестность, то дальше рассматривается каждая точка этой плотной окрестности, и вокруг неё также строится $\varepsilon$-окрестность, и так пока не будет достигнута граница некоторого множества объектов. 

Хорошей аналогией может служить лес: один лес - это один кластер, через опушку, второй лес, - другой кластер, мы находимся в лесу. Смотрим, в нашей окрестности деревьев много, это значит, что мы в корневой точке находимся, и дальше мы идём, пока не выйдем на опушку леса, там мы окажемся в граничной точке - она уже не корневая, вокруг деревьев меньше. А где-то могут расти отдельно стоящие деревья - это шумовые выбросы. И вот так ходим по лесу, пока его весь не обойдём, и как только мы обошли весь лес, назовем его кластером. После чего случайно выбираем новое дерево и начинаем строить другой кластер.

Формализуем алгоритм в виде псевдокода:\\
\begin{tabularx}{\linewidth}{lX}
\textbf{вход:} выборка $X^l - \{x_1,...,x_l\}$; параметры $\varepsilon$ и $m$\\
\textbf{выход:} разбиение выборки на кластеры и шумовые выбросы;\\\hspace*{7mm}\hspace*{9mm}$U := X^l$ - не помеченные точки, $a := 0$\\
\textbf{пока} в выборке есть непомеченные точки, $U \neq \emptyset$:\\
\hspace*{7mm} взять случайную точку $x \in U$; \\
\hspace*{7mm} \textbf{если} $\abs{U_\varepsilon (x)} < m$ \textbf{то} \\
\hspace*{7mm}\hspace*{7mm} пометить $x$ как, возможно, шумовой;\\
\hspace*{7mm}\textbf{иначе} \\
\hspace*{7mm}\hspace*{7mm} создать новый кластер: $K:=U_\varepsilon (x); \; a:=a+1;$ \\
\hspace*{7mm}\hspace*{7mm} \textbf{для всех} $x' \in K$, не помеченных или шумовых \\
\hspace*{7mm}\hspace*{7mm}\hspace*{7mm} \textbf{если} $\abs{U_\varepsilon (x')} \geq m$,  \textbf{то} $K := K \cup U_\varepsilon (x')$; \\
\hspace*{7mm}\hspace*{7mm}\hspace*{7mm} \textbf{иначе} поментить $x'$ как граничный кластера $K$;\\
\hspace*{7mm}\hspace*{7mm} $a_j := a$ для всех $x_i \in K$;\\
\hspace*{7mm}\hspace*{7mm} $U := U \textbackslash K$;\\
\vspace{5mm}
\end{tabularx}

В таком виде алгоритм обладает следующими \textbf{свойствами}:
\begin{itemize}
    \item быстрая кластеризация больших данных: \\$O(l^2)$ в худшем случае, \\ $O(l \mathrm{ln} l)$ при эффективной реализации $U_\varepsilon (x)$;
    \item кластеры произвольной формы
    \item деление объектов на корневые, граничные, шумовые.
\end{itemize}

При этом важно понимать, что граничные объекты не выстраивают в точности границу каждого кластера. Практически это означает, что не стоит всерьез рассматривать граничные объекты, в отличие от шумовых, которые действительно можно в дальнейшем анализировать.

\subsection{Примечание о HDBSCAN} 
От гиперпараметра $\varepsilon$ можно избавиться, используя дивизивную кластеризацию. Такая модификация называется HDBSCAN. Его суть проста: необходимо построить дендрограмму, где по $Оу$ будет отложен $\varepsilon$ (на рис.\ref{fig:hdbdendro} снизу distance). Так мы сможем явно видеть вложенные кластеры. Алгоритм затем сам вычисляет оптимальное количество кластеров на основе метрики "стабильности кластеров".

\begin{figure}[h!]
    \centering
    \includegraphics[width=0.6\linewidth]{png/hdbscan_dendrogramm.png}
    \caption{К примечанию о HDBSCAN}
    \label{fig:hdbdendro}
\end{figure}
\subsection{Задачи}
\textbf{Задача 1.}

\textbf{Условие.} Применить DBSCAN для выборки из таблицы с $m=4,\;\varepsilon=1.9$. Метрика евклидова.

\begin{center}
\begin{tabular}{ |c|c|c| } 
 \hline
 P1(3,7) & P5(7,3) & P9(3,3) \\ 
 P2(4,6) & P6(6,2) & P10(2,6) \\ 
 P3(5,5) & P7(7,2) & P11(3,5) \\ 
 P4(6,4) & P8(8,4) & P12(2,4) \\ 
 \hline
\end{tabular}
\end{center}

\textbf{Решение.}
Запишем матрицу, составленную из соответственных расстояний между точками выборки:
\begin{center}
\begin{tabular}{ |c|c|c|c|c|c|c|c|c|c|c|c|c|} 
 \hline
dot & P1 & P2 & P3 & P4 & P5 & P6 & P7 & P8 & P9 & P10 & P11 & P12 \\ \hline
P1 & 0 &  &  &  &  &  &  &  &  &  &  &   \\ \hline
P2 & 1.41 & 0 &  &  &  &  &  &  &  &  &  &   \\ \hline
P3 & 2.83 & 1.41 & 0 &  &  &  &  &  &  &  &  &   \\ \hline
P4 & 4.24 & 2.83 & 1.41 & 0 &  &  &  &  &  &  &  &   \\ \hline
P5 & 5.66 & 4.24 & 2.83 & 1.41 & 0 &  &  &  &  &  &  &   \\ \hline
P6 & 5.83 & 4.47 & 3.16 & 2.00 & 1.41 & 0 &  &  &  &  &  &   \\ \hline
P7 & 6.40 & 5.00 & 3.61 & 2.24 & 1.00 & 1.00 & 0 &  &  &  &  &   \\ \hline
P8 & 5.83 & 4.47 & 3.16 & 2.00 & 1.41 & 2.83 & 2.24 & 0 &  &  &  &   \\ \hline
P9 & 4.00 & 3.16 & 2.83 & 3.16 & 4.00 & 3.16 & 4.12 & 5.10 & 0 &  &  &   \\ \hline
P10& 1.41 & 2.00 & 3.16 & 4.47 & 5.83 & 5.83 & 5.66 & 6.40 & 6.32 & 0 &  &   \\ \hline
P11& 2.00 & 1.41 & 2.00 & 3.16 & 4.47 & 4.24 & 5.00 & 5.10 & 2.00 & 1.41 & 0 &   \\ \hline
P12& 2.83 & 3.16 & 4.00 & 5.10 & 4.47 & 5.39 & 6.00 & 1.41 & 2.00 & 2.00 & 1.41 & 0  \\ \hline
\end{tabular}
\end{center}
Сравнивая значения в каждом столбце матрицы с $\varepsilon$ и отбирая те, что меньше этого значения, находим окрестности каждой точки.

\begin{center}
\begin{tabular}{ |c|c| } 
 \hline
 точка & окрестность \\\hline
 P1 & P2, P10\\ 
 P2 & P1, P3, P11\\ 
 P3 & P2, P4\\ 
 P4 & P3, P5\\
 P5 & P4, P6, P7, P8\\
 P6 & P5, P7\\
 P7 & P5, P6\\
 P8 & P5\\
 P9 & P12\\
 P10 & P1, P11\\
 P11 & P2, P10, P12\\
 P12 & P9, P11\\
 \hline
\end{tabular}
\end{center}

Если в окрестности больше $m=4$ точек (включая ее саму), то отнесем эту точку к корневой, иначе - к шумовой.

\begin{center}
\begin{tabular}{ |c|c| } 
 \hline
 точка & тип \\\hline
 P1 & шум\\ 
 P2 & корневая\\ 
 P3 & шум\\ 
 P4 & шум\\
 P5 & корневая\\
 P6 & шум\\
 P7 & шум\\
 P8 & шум\\
 P9 & шум\\
 P10 & шум\\
 P11 & корневая\\
 P12 & шум\\
 \hline
\end{tabular}
\end{center}

Уточним классификацию, учтя граничные точки, т.е. точки, лежащие в окрестности корневых, но при этом не являющимися корневыми:
\begin{center}
\begin{tabular}{ |c|c| } 
 \hline
 точка & тип \\\hline
 P1 & граничная\\ 
 P2 & корневая\\ 
 P3 & граничная\\ 
 P4 & граничная\\
 P5 & корневая\\
 P6 & граничная\\
 P7 & граничная\\
 P8 & граничная\\
 P9 & шум\\
 P10 & граничная\\
 P11 & корневая\\
 P12 & граничная\\
 \hline
\end{tabular}
\end{center}

К первому кластеру отнесем окрестность корневой точки 2, причем в ее окрестности находится еще одна корневая точка 11, так что отнесем и ее окрестность к первому кластеру. Ко второму кластеру отнесем корневую точку 5 и ее окрестность. Осталась лишь одна точка P9, которая не относится ни к какому кластеру и является шумовой.
\begin{figure}[h!]
    \centering
    \includegraphics[width=0.7\linewidth]{png/task1dbs_plot.png}
    \caption{Кластеризация в задаче 1}
    \label{fig:task1dbs}
\end{figure}

\begin{minipage}{.5\textwidth}
\textbf{Задача 2.}\\
\textbf{Условие.}
  Сравните результаты кластеризации с помощью k-means и с помощью DBSCAN и объясните их.\\
\textbf{Решение.}
Объяснение различий:
\begin{itemize}
\item \textit{Форма кластера}:
K-средние: стремится найти сферические или выпуклые кластеры. Предполагается, что кластеры изотропны (однородны во всех направлениях) и имеют схожий размер.
DBSCAN: может обнаруживать кластеры произвольной формы и размера. Не делает предположений о форме кластеров.
\item \textit{Обработка шума}:
K-средние: плохо справляется с шумом. Точки шума могут быть назначены кластерам, что может повлиять на центры кластеров.
DBSCAN: может идентифицировать и маркировать точки шума, которые не назначены ни одному кластеру.
\end{itemize}
\end{minipage}% This must go next to `\end{minipage}`
\begin{minipage}{.4\textwidth}
      \includegraphics[width=0.95\linewidth]{png/task2dbs_plot.png}
\end{minipage}
\begin{itemize}
\item \textit{Плотность кластера}:
K-средние: не учитывает плотность точек. Каждый кластер представлен центроидом.
DBSCAN: учитывает плотность точек. Кластеры формируются на основе плотности точек в окрестности.
\item \textit{Чувствительность параметров}:
K-средние: требует предварительного указания количества кластеров (K), так что, если если заранее указать 3 кластера, то алгоритм и найдет три кластера, даже если он всего один, как на последней паре картинок.
\end{itemize}

\textbf{Задача 3.}\\
\textbf{Предисловие.}
При решении задачи 1 использовалась матрица, состоящая из расстояний между парами точек (\textit{матрица смежности}). Понятием, противоположным расстоянию, является понятие сходства между объектами. Неотрицательная вещественная функция $S(x_i,x_j) = S_{ij}$ называется \textit{мерой сходства}, если:
\begin{itemize}
    \item $0 \leq S(x_i,x_j) < 1$, для $x_i \neq x_j$
    \item $S(x_i,x_j)=1$
    \item $S(x_i,x_j)=S(x_j,x_i)$
\end{itemize}
Пары значений мер сходства можно объединить в \textit{матрицу сходства} $S$, симметричную и единичной диагональю.
\textbf{Условие.}
Применить DBSCAN с пороговым значением \textit{меры сходства} 0.8 и $m = 2$ и заданной матрицей сходства между точками выборки:

\begin{center}
\begin{tabular}{ |c|c|c|c|c|c|} 
 \hline
dot & P1 & P2 & P3 & P4 & P5  \\ \hline
P1 & 1.0 &  &  &  &     \\ \hline
P2 & 0.10 & 1.0 &  &  &  \\ \hline
P3 & 0.41 & 0.64& 1.0 &  & \\ \hline
P4 & 0.55 & 0.47 & 0.44 & 1.0 & \\ \hline
P5 & 0.35 & 0.98 & 0.85 & 0.76 & 1.0 \\ \hline
\end{tabular}
\end{center}

Сравнивая значения в каждом столбце матрицы с $\varepsilon$ и выбирая те точки, для которых значение сходства выше, чем порог, формируем окрестности всех точек.

\begin{center}
\begin{tabular}{ |c|c| } 
 \hline
 точка & окрестность \\\hline
 P1 & -\\ 
 P2 & P5\\ 
 P3 & P5\\ 
 P4 & -\\
 P5 & P2, P3\\
 \hline
\end{tabular}
\end{center}

Если в окрестности больше $m=2$ точек (включая ее саму), то отнесем эту точку к корневой, иначе - к шумовой.

\begin{center}
\begin{tabular}{ |c|c| } 
 \hline
 точка & тип \\\hline
 P1 & шум\\ 
 P2 & корневая\\ 
 P3 & корневая\\ 
 P4 & шум\\
 P5 & корневая\\
 \hline
\end{tabular}
\end{center}

Уточнение классификации, путем учитывания граничных точек, т.е. точек, лежащие в окрестности корневых, но при этом не являющимися корневыми, ничего не дает, т.к. в окрестности точек, определенных как шумовые вообще нет других точек, так что они действительно являются шумом.

К первому кластеру отнесем окрестность корневой точки P2, причем в ее окрестности находятся еще краевая точка P5, так что отнесем ее к этому же кластеру. В окрестности точки P5 помимо уже классифицированной P2 находится еще корневая точка P3, которую также отнесем к первому кластеру. Остальные точки классифицированы как шумовые. Таким образом в данной задаче всего один кластер, состоящий из точек P2, P3, P5. 





\end{document}

\end{document}