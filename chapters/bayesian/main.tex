\section*{Байесовские методы классификации}
Решаем задачу классификации. Пусть $A = A_1 \times \ldots \times A_m$ --- пространство объектов, $B$ --- конечное множество классов. Предположим, что объекты $(x, y) \in A \times B$ независимо выбираются из какого-то неизвестного распределения с обобщённой плотностью распределения $p(x, y)$. Пусть $(X, Y)$ --- случайная величина на $A \times B$ с таким распределением. Хотим по $x$ находить его наиболее вероятный класс, то есть класс $y$, максимизирующий $P(Y=y|X=x)$. По формуле Байеса $P(y|x)=\frac{P(y)p(x|y)}{p(x)}$, поэтому максимизация $P(y|x)$ равносильна максимизации $P(y)p(x|y)$. Таким образом, задача сводится к восстанавлению дискретного априорного распределения $P(y)$ и восстановлению условного распределения $p(x|y)$.

Обычно предполагается, что $Y$ имеет произвольное категориальное распределение на $B$, то есть что о распределении $Y$ нет никакой информации, кроме множества принимаемых значений. В этом случае можно аналитически найти оценку на параметры распределения $P(Y=b_i)$ методом максимального правдоподобия.

\textbf{Задача 1.} Пусть $N$ --- количество элементов в выборке. Для любого $b \in B$ обозначим через $N_b$ --- количество элементов, для которых $y=b$ и через $\overline{p_b}$ --- частоту, с которой $y$ принимает значение $b$, то есть $\overline{p_b} = \frac{N_b}{N}$. Докажите, что $\overline{p_b}$ --- оценка максимального правдоподобия вероятностей $P(Y=b)$. \\
\textit{Указание: перейдите к максимизации логарифма правдоподобия и воспользуйтесь неравенством Гиббса.}


\subsection*{Наивный Байесовский классификатор}

Наивный Байесовский классификатор делает предположение, что признаки независимы в совокупности при условии классов, то есть для любого $k$, любых $i_1 < \ldots < i_k$ и любых $x_1, \ldots, x_k, y$ выполнено $p(X_{i_1}=x_1, \ldots, X_{i_k} = x_k|Y=y)=p(X_{i_1}=x_1|Y=y)\cdot\ldots\cdot p(X_{i_k}=x_k|Y=y)$. Отметим, что треубется именно независимость $X_i$ при условии $Y$, а не просто независимость $X_i$.

\textbf{Задача 2.} \\
a) Приведите пример совместного распределения бернуллиевских случайных величин $X_1, X_2, Y$ при котором $X_1$ и $X_2$ независимы, но не независимы при условии $Y$. \\
b) Приведите пример совместного распределения бернуллиевских случайных величин $X_1, X_2, Y$ при котором $X_1$ и $X_2$ независимы при условии $Y$, но не независимы.

Для каждого класса $b \in B$ и для каждого признака решается одномерная задача восстановления плотности $P(X_i|Y=b)$. Таким образом, предположение об условной независимости позволяет свести сложную задачу восстановления плотности многомерного распределения к более простой задаче восстановления плотности одномерного распределения.

Рассмотрим случай, когда предполагается, что условные расперделения признаков при условии класса берутся из какого-то экспоненциального семейства распределений, то есть $p(x|y) = \exp \left(\frac{\theta_y x-c(\theta_y)}{\phi_y} + h(x, \phi_y) \right)$, где $\theta_y, \phi_y$ --- параметры распределения. Параметры распределения оцениваем метода максимального правдоподобия, то есть $(\overline{\theta_y}, \overline{\phi_y}) = \text{argmax}_{\theta, \phi}\left(\sum_{i=1}^{N} \frac{\theta_yx^i-c(\theta_y)}{\phi_y} + h(x^i, \phi_y) \right)$.
Для многих распределений эта задача оптимизации решается аналитически.

В итоге после восстановления параметров получаем формулу для оценки вероятности класса
$$P(Y=y|X=x) = \frac{P(Y=y)p(X=x|Y=y)}{p(X=x)} = $$ $$ = \exp \left( \sum_{j=1}^{m} \frac{\overline{\theta_{yj}}}{\overline{\phi_{yj}}}x_j + \sum_{j=1}^{m}h(x_j, \overline{\phi_{yj}}) - \sum_{j=1}^{m}\frac{c_j(\overline{\theta_{yj}})}{\overline{\varphi_{yj}}} + \ln \overline{P(Y=y)} - \ln p(X=x) \right).$$

В случае, если $\overline{\varphi_{yj}}$ не зависит от $y$, то $\sum_{j=1}^{m}h(x_j, \overline{\phi_{yj}})$ и $\ln p(X=x)$ не зависят от $y$, поэтому максимизация вероятности класса эквивалентна максимизации $\sum_{j=1}^{m}w_{yj}x_j + w_{y0}$, где $w_{yj}=\frac{\overline{\theta_{yj}}}{\overline{\phi_{yj}}}$, $w_{y0}=\ln \overline{P(Y=y)} - \sum_{j=1}^{m}\frac{c_j(\overline{\theta_{yj}})}{\overline{\varphi_{yj}}}.$ Таким образом, в этом случае наивный байесовский классификатор строит линейную разделяющую поверхность.

\textbf{Задача 3.} Проверьте, что если все признаки бинарные, то наивный байесовский классификатор с $2$ классами эквивалентен логистической регрессии с фиксированными весами и найдите эти веса.