\section*{Нестандартные функции потерь. Метод наименьших модулей. Квант\'{и}льная регрессия.}

Квадратичная функция потерь обычно дает хорошие результаты, является удобной в использовании, а потому и применяется чаще всего. Но в некоторых ситуациях она все же неприменима, и приходится использовать нестандартные функции потерь.

\subsection*{Метод наименьших модулей.}

Здесь и далее будем использовать следующие обозначения: $\ell$ - количество объектов в тренировочной выборке, $x_i$ ($i \in \left\{1, \dotsc, \ell \right\}$) - вектор признаков, $y_i$ ($i \in \left\{1, \dotsc, \ell \right\}$) - таргет, $a$ - модель, $\varepsilon_i := a(x_i) - y_i$ ($i \in \left\{1, \dotsc, \ell \right\}$).

Для стандартной квадратичной функции потерь $\mathscr{L}(\varepsilon) = \varepsilon^2$ задача будет выглядеть так:
$$\frac{1}{\ell}\sum\limits_{i=1}^\ell\left(a(x_i) - y_i\right)^2 \longrightarrow \min\limits_{a}.$$

Заменим функцию потерь на $\mathscr{L}(\varepsilon) = |\varepsilon|$.
Задача станет выглядеть так:
$$\frac{1}{\ell}\sum\limits_{i=1}^\ell\left|a(x_i) - y_i\right| \longrightarrow \min\limits_{a}.$$
Данный метод называется \textit{методом наименьших модулей}.

В какой ситуации такой подход может оказаться полезным? Рассмотрим ситуацию,  когда наша модель - это константа, то есть она вообще не зависит от признаков. Для квадратичной функции потерь получим:
$$\frac{1}{\ell}\sum\limits_{i=1}^\ell\left(a - y_i\right)^2 \longrightarrow \min\limits_{a}.$$
Здесь можно посчитать ответ аналитически, оптимальной константой $a$ будет $a = \frac{1}{\ell}\sum\limits_{i=1}^\ell y_i$, то есть среднее арифметическое таргетов. Это плохая оценка, если, например, в нашей выборке присутствуют выбросы, или распределение ошибок имеет тяжёлые <<хвосты>>.

А в случае использования метода наименьших модулей задача будет выглядеть так:
$$\frac{1}{\ell}\sum\limits_{i=1}^\ell\left|a - y_i\right| \longrightarrow \min\limits_{a}.$$
Здесь опять же можно посчитать ответ аналитически, оптимальным $a$ будет $a = \median\left\{ y_1, \dotsc, y_\ell \right\} = y_{(\ell / 2)}$ (серединный член вариационного ряда). Эта оценка хороша тем, что устойчива к выбросам, хорошо работает для распределений ошибок с тяжелыми <<хвостами>>.

Таким образом, использование метода наименьших модулей в некоторых ситуациях помогает бороться с выбросами.

\subsection*{Квант\'{и}льная регрессия.}

Метод наименьших модулей можно обобщить. Давайте по-разному штрафовать отрицательные и положительные ошибки. Т.е. будем рассматривать функцию потерь вида
$$\mathscr{L}(\varepsilon) = \begin{cases}
    C_+|\varepsilon|, \varepsilon \geq 0,\\
    C_-|\varepsilon|, \varepsilon < 0.
\end{cases}$$

\begin{figure}[h]
    \centering
    \includegraphics{chapters/nonstandart_error/images/ФПКР.png}
\end{figure}

Опять же рассмотрим случай, когда модель не зависит от признаков, то есть является константой:
$$\frac{1}{\ell}\sum\limits_{i=1}^\ell\mathscr{L}\left(a - y_i\right) \longrightarrow \min\limits_{a}.$$
Решение данной задачи опять же можно получить аналитически, оптимальным $a$ будет $a = y_{(q)}$ ($q$-тый член вариационного ряда), где $q = \frac{\ell C_-}{C_- + C_+}$.
Именно поэтому используемый метод называется \textit{методом квант\'{и}льной регрессии}.

Рассмотрим также еще один случай, когда квантильная регрессия оказывается хорошим вариантом с точки зрения решения задачи оптимизации, а именно случай линейной модели: $a(x) = \left\langle x, w \right\rangle$, где $w$ - вектор весов.

Сделаем замену переменных $\varepsilon_i^+ = (a(x_i) - y_i)_+$, $\varepsilon_i^- = (a(x_i) - y_i)_-$. Тогда наша задача будет иметь вид
$$\begin{cases}
    \frac{1}{\ell}\sum_{i=1}^\ell C_+\varepsilon_i^+ + C_-\varepsilon_i^- \longrightarrow \min\limits_{w},\\
    \left\langle w, x_i \right\rangle - y_i = \varepsilon_i^+ - \varepsilon_i^-, i \in \left\{ 1 , \dotsc, \ell \right\}.
\end{cases}$$

Это задача линейного программирования, для решения которой существует масса способов.

\newpage

\subsection*{Задачи.}

\subsubsection*{Задача 1.}

Предположим, у Вас есть датасет с данными о жителях некоторой страны Южной или Центральной Африки, а также данные об их среднем ежедневном доходе, и Вы хотите научиться предсказывать этот доход по значениям рассматриваемых признаков. Полученная модель в последующем будет использоваться для составления плана по оказанию гуманитарной помощи населению. Что использовать предпочтительнее: квадратичную функцию потерь или функцию потерь квант\'{и}льной регрессии? Почему? Если использовать предпочтительнее функцию потерь квант\'{и}льной регрессии, то каким должно быть соотношение параметров $C_+$ и $C_-$ и почему?

\begin{solution}
    В условии задачи не зря указано, из какого региона у нас страна. Б\'{о}льшая часть населения в ней, скорее всего, крайне бедна, при этом есть очень незначительное количество сверхбогатых людей, т.е., в нашей терминологии, выбросов. Поэтому функция потерь квант\'{и}льной регрессии более предпочтительна, чем квадратичная функция потерь.

    Теперь обратим внимание на то, как будет использована модель, чтобы понять, какая ошибка для нас наиболее страшна: положительная или отрицательная. Заметим, что если мы предсказали доход человека больше реального, т.е. ошибка положительна, то, скорее всего, на него будет выделено меньше помощи. Это кажется более страшным, чем ситуация, в которой мы предсказали доход человека меньше реального, и дали ему чуть больше помощи. Поэтому штрафовать положительные ошибки стоит сильнее, чем отрицательные, то есть стоит выбрать $C_+$ и $C_-$ так, чтобы $C_+/C_- > 1$.
\end{solution}

\subsubsection*{Задача 2.}

Предположим, что у в природе некоторый таргет действительно линейно зависит от вектора признаков, но вектор таргетов, который у нас есть, зашумлен ошбиками, причем ошибки эти приходят из симметричного распределения Коши. Почему использование квадратичной функции потерь в данном случае будет крайне нежелательным? А если ошибки приходит из распределения $\mathcal{N}(a, \sigma^2)$, где $a \neq 0$?

\begin{solution}
    Из теории вероятностей и математической статистики известно, что у распределения Коши тяжелые <<хвосты>>, из-за чего у него даже нет матожидания. Именно поэтому квадратичная функция потерь, очень сильно штрафующая за большие ошибки, здесь не подходит. А вот метод наименьших модулей подхолит лучше.

    Также можно заметить, что квадратичная функция потерь одинаково штрафует положительные и отрицательные ошибки, то есть неявно подразумевается симметричность распределения. В общем случае это не так. Квант\'{и}льная регрессия позволяет более гибко работать с несимметричными распределениями.
\end{solution}

\subsubsection*{Задача 3.}

Предположим, что Вы тестируете новое лекарство на мышах. Вы хотите понять, какая доза лекарства оптимальна: уже вылечивает мышь, но все еще не дает негативных побочных эффектов. Известно, что каждый новый побочный эффект проявляется при превышении некоторого примерно одинакового для всех мышей порога передозировки и действует все сильнее и сильнее при дальнейшем превышении этого порога. Также известно, что оптимум не единственен, а достигается на некотором отрезке, длина которого вам тоже заранее известна. Наконец, ниже некоторого порога лекарство действует все хуже и хуже. У Вас есть некоторые данные о мышах, а также таргеты - максимальные оптимальные дозы лекарств. Предложите модификацию функции потерь квант\'{и}льной регрессии, оптимальную для данной задачи.

\begin{solution}
    Основная идея квант\'{и}льной регрессии по сравнению с методом наименьших модулей состоит в том, что мы по-разному штрафуем положительные и отрицательные ошибки. Давайте разовьем эту идею, и будем по-разному штрафовать ошибки в зависимости от того, в какую часть числовой прямой мы попали.

    Так, для нашей задачи, пусть новые побочные эффекты проявляются при превышении дозировки на $a_1, \dotsc, a_n$ у.е., где $0= a_1 < \ldots < a_n$. А длина оптимального отрезка дозировки равна $b$, Тогда можно взять такую функцию потерь:
    $$\mathscr{L}(\varepsilon) = \begin{cases}
        -v_b\varepsilon, \varepsilon < -b,\\
        0, \varepsilon \in [-b; a_1],\\
        v_{a_1}\varepsilon, \varepsilon \in (a_1; a_2],\\
        (v_{a_1} + v_{a_2})\varepsilon, \varepsilon \in (a_2; a_3],\\
        \vdots\\
        (v_{a_1} + \dotsb + v_{a_{n - 1}})\varepsilon, \varepsilon \in (a_{n-1}; a_n],\\
        (v_{a_1} + \dotsb + v_{a_n})\varepsilon, \varepsilon > a_n,
    \end{cases}$$
    где $v_b, v_{a_1}, \dotsc, v_{a_n}$ - скорости роста соответствующих проблем.
\end{solution}


\section*{Энтропийные функции потерь. Перекрёстная энтропия.}

В задачах классификации (бинарной или многоклассовой) очень распространено использование энтропийных функций потерь, в первую очередь перекрёстной энтропии. Эти функции потерь тесно связаны с понятиями теории вероятностей и информационной теории, в частности с понятием дивергенции Кульбака–Лейблера и энтропией Шеннона.

Основная идея: мы хотим не просто предсказать «класс», но и оценить распределение вероятностей классов. Пусть у нас есть:

\begin{itemize}
    \item Набор объектов: $(x_1, y_1), \dots, (x_\ell, y_\ell)$, где $x_i$ – вектор признаков, а $y_i$ – соответствующий истинный класс объекта (для простоты – номер класса или унитарный вектор-«one-hot»).
    \item Модель $a$, выдающая оценку распределения вероятностей по классам для входа $x_i$. Обозначим предсказанную моделью вероятность класса $k$ как $p_{ik} = a_k(x_i)$, где $\sum_k p_{ik} = 1$.
\end{itemize}

\subsection*{Бинарная классификация}

В случае двух классов, обозначим целевой класс как $y_i \in \{0,1\}$ и предсказанную моделью вероятность класса 1 для объекта $i$ как $p_i = a_1(x_i)$. Тогда функция потерь на одном объекте может быть записана как (перекрёстная энтропия для бинарного случая):

$$
\mathcal{L}(y_i, p_i) = -\left[y_i\log(p_i) + (1 - y_i)\log(1 - p_i)\right].
$$

Эта функция потерь равна отрицательному логарифму правдоподобия при условии, что $y_i$ берётся из Бернуллиевского распределения с параметром $p_i$. Минимизация этой функции потерь равносильна максимизации правдоподобия.

\subsection*{Многоклассовая классификация}

Пусть класс $y_i$ закодирован в one-hot формате как вектор $Y_i = (y_{i1}, \dots, y_{iK})$, где $y_{ik} = 1$, если объект относится к классу $k$, и 0 в противном случае. Предсказание модели есть вероятностный вектор $P_i = (p_{i1}, \dots, p_{iK})$. Тогда перекрёстная энтропия:

$$
\mathcal{L}(Y_i, P_i) = -\sum_{k=1}^{K} y_{ik}\log(p_{ik}).
$$

Оптимизируя по параметрам модели, мы стремимся сделать предсказанное распределение вероятностей $P_i$ как можно более «острым» и совпадающим с истинным распределением мишени $Y_i$ (которое в обучающих данных детерминировано и задаётся one-hot вектором).

\subsection*{Связь с KL-дивергенцией}

Перекрёстная энтропия между истинным распределением $Y_i$ и предсказанным $P_i$ связана с дивергенцией Кульбака–Лейблера:

$$
H(Y_i, P_i) = H(Y_i) + D_{\text{KL}}(Y_i \| P_i),
$$

где $H(Y_i)$ – энтропия истинного распределения (константа в рамках оптимизации), а $D_{\text{KL}}(Y_i \| P_i)$ – дивергенция Кульбака–Лейблера, которая всегда неотрицательна. Минимизируя перекрёстную энтропию, мы минимизируем $D_{\text{KL}}$, что приводит к более точному приближению истинного распределения предсказанным.

\subsection*{Модификации}

\begin{itemize}
    \item \textbf{Взвешенная перекрёстная энтропия}: для решения проблем несбалансированных классов можно вводить веса $w_k$ для каждого класса:
    $$
    \mathcal{L}(Y_i, P_i) = -\sum_{k=1}^{K} w_k y_{ik}\log(p_{ik}).
    $$
    
    \item \textbf{Label smoothing}: позволяет сгладить «жесткие» one-hot лейблы, заменяя значение 1 на $1-\alpha$ и распределяя оставшуюся массу $\alpha$ равномерно по остальным классам. Это снижает перенакрой и делает модель более устойчивой.
\end{itemize}

\newpage

\section*{Задачи}

\subsubsection*{Задача 1 (Бинарная классификация и правдоподобие)}

Пусть у нас есть задача бинарной классификации: $y_i \in \{0,1\}$, и модель $a(x_i)$, дающая оценку вероятности класса 1 для $x_i$. Предположим, что истинное распределение метки $y_i$ для данного $x_i$ – это бернуллиевское распределение с параметром $p_i = a(x_i)$. Покажите, что минимизация средней перекрёстной энтропии

$$
\frac{1}{\ell}\sum_{i=1}^{\ell} \left[-y_i\log(p_i) - (1 - y_i)\log(1 - p_i)\right]
$$

эквивалентна максимизации правдоподобия выборки. Объясните, почему это даёт статистически обоснованную функцию потерь.

\begin{solution}
Правдоподобие выборки при условии независимости объектов есть:

$$
L = \prod_{i=1}^{\ell} p_i^{y_i}(1 - p_i)^{1 - y_i}.
$$

Взяв логарифм, получим:

$$
\log L = \sum_{i=1}^{\ell} [y_i\log(p_i) + (1 - y_i)\log(1 - p_i)].
$$

Максимизация $\log L$ по параметрам модели эквивалентна минимизации

$$
-\log L = -\sum_{i=1}^{\ell} [y_i\log(p_i) + (1 - y_i)\log(1 - p_i)],
$$

что есть сумма (или среднее) перекрёстной энтропии. Таким образом, минимизация перекрёстной энтропии совпадает с максимизацией правдоподобия. Это придаёт функции потерь статистическое обоснование: мы получаем состоятельную оценку параметров модели при верном специфицировании вероятностной модели.
\end{solution}

\subsubsection*{Задача 2 (Небаланс классов)}

Предположим, что у вас есть задача многоклассовой классификации с сильно несбалансированными классами. Один из классов встречается существенно реже других. Объясните, как можно модифицировать функцию перекрёстной энтропии, чтобы придать более высокий штраф за неправильную классификацию редкого класса. Предложите конкретную формулу и аргументируйте её использование.

\begin{solution}
Стандартная перекрёстная энтропия для многоклассовой задачи:

$$
\mathcal{L}(Y_i, P_i) = -\sum_{k=1}^K y_{ik}\log(p_{ik}).
$$

Если класс $r$ встречается редко, мы можем ввести для него повышающий вес $w_r > 1$. Тогда функция потерь модифицируется так:

$$
\mathcal{L}(Y_i, P_i) = -\sum_{k=1}^K w_k y_{ik}\log(p_{ik}),
$$

где $w_k = 1$ для всех «частых» классов, а $w_r > 1$ для редкого класса. Это увеличивает штраф за ошибку на редком классе и стимулирует модель уделять ему больше внимания. В итоге модель будет «стараться» точнее предсказывать редкий класс, жертвуя немного точностью на других, более частых классах, что зачастую улучшает общую полезность модели при решении практических задач с несбалансированными данными.
\end{solution}

\subsubsection*{Задача 3 (Label Smoothing)}

Предположим, что у вас есть задача многоклассовой классификации с $K$ классами. Истинные метки заданы в виде one-hot векторов. Вы подозреваете, что модель может слишком «уверенно» переобучаться, пытаясь подогнать вероятности к очень жёсткому распределению (где истинный класс имеет вероятность 1, а остальные 0). Предложите модификацию перекрёстной энтропии (label smoothing), объясните, в чём она заключается и как именно изменится функция потерь.

\begin{solution}
Идея label smoothing состоит в том, чтобы заменить истинный one-hot вектор $Y_i$ на более «сглаженный» вектор $Y_i'$. Пусть $\alpha$ – небольшой положительный параметр сглаживания (например, 0.1). Тогда:

\begin{itemize}
    \item Для истинного класса $c_i$ объекта $i$ мы присваиваем $y_{ic_i}' = 1 - \alpha$.
    \item Для всех остальных классов $k \neq c_i$ мы присваиваем $y_{ik}' = \frac{\alpha}{K-1}$.
\end{itemize}

Таким образом, истинный вектор $(0,\dots,0,1,0,\dots,0)$ превращается в вектор, где истинный класс имеет вероятность чуть меньше 1, а остальные классы получают небольшую ненулевую вероятность.

Новая функция потерь будет выглядеть так:

$$
\mathcal{L}(Y_i', P_i) = -\sum_{k=1}^K y_{ik}' \log(p_{ik}).
$$

Поскольку $Y_i'$ теперь не является «жёстким» one-hot вектором, модель не будет излишне стремиться предсказывать для истинного класса вероятность ровно 1, что снижает риск переобучения и делает распределение прогнозов более гладким и устойчивым.
\end{solution}
