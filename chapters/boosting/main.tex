\section*{Гармонический бустинг (Harmonic Boosting)}

Гармонический бустинг (Harmonic Boosting) представляет собой ансамблевый метод, целью которого является минимизация вклада моделей с высокой ошибкой путём взвешивания их предсказаний на основе гармонического среднего. Этот подход улучшает устойчивость ансамбля и снижает риск ухудшения качества из-за наличия "шумных" моделей.

\subsection*{Основная идея}

Гармоническое среднее даёт больший вес точным моделям и минимизирует влияние слабых или ошибочных предсказаний. Для \( n \) моделей ансамбля итоговое предсказание классификации вычисляется как:
\[
\hat{y} = \arg\max_{k} \left( \frac{n}{\sum_{i=1}^n \frac{1}{P_k(y_i)}} \right),
\]
где \( P_k(y_i) \) — вероятность принадлежности к классу \( k \), предсказанная \( i \)-й моделью.

Для регрессии итоговое предсказание:
\[
\hat{y} = \frac{n}{\sum_{i=1}^n \frac{1}{y_i}},
\]
где \( y_i \) — результат \( i \)-й модели.

\subsection*{Математическое обоснование}

Рассмотрим задачу регрессии. Пусть \( y_i \) — предсказание \( i \)-й модели, а \( e_i \) — её ошибка. Общая ошибка ансамбля определяется как:
\[
E = \frac{n}{\sum_{i=1}^n \frac{1}{e_i}}.
\]
Гармоническое среднее минимизирует \( E \), поскольку больший вес получают модели с меньшей ошибкой. Это свойство обеспечивает устойчивость ансамбля и снижает влияние "шумных" моделей.

Для классификации гармоническое среднее вероятностей используется в логарифмическом масштабе:
\[
\log P(C_k) = -\sum_{i=1}^n \frac{1}{\log P_k(y_i)}.
\]
Такой подход позволяет смягчить влияние моделей с крайне низкой вероятностью.

\subsection*{Алгоритм гармонического бустинга}

\textbf{Вход:}
\begin{itemize}
    \item Набор данных \( \mathcal{D} = \{(x_i, y_i)\}_{i=1}^m \),
    \item Количество слабых моделей \( T \),
    \item Функция потерь \( L(y, \hat{y}) \).
\end{itemize}

\textbf{Шаги алгоритма:}
\begin{enumerate}
    \item Инициализация ансамбля с первой моделью \( h_1 \), обученной на \( \mathcal{D} \).
    \item Вычисление весов для объектов в \( \mathcal{D} \) на основе обратной ошибки:
    \[
    w_j^{(t)} = \frac{1}{e_j^{(t)}},
    \]
    где \( e_j^{(t)} = L(y_j, h_t(x_j)) \).
    \item Построение новой модели \( h_{t+1} \) с учётом весов \( w_j^{(t)} \).
    \item Итоговое предсказание ансамбля:
    \[
    \hat{y} = \frac{n}{\sum_{i=1}^T \frac{1}{h_i(x)}}.
    \]
\end{enumerate}

\subsection*{Сравнение с другими методами бустинга}

В отличие от AdaBoost и градиентного бустинга, где обновление весов основывается на увеличении влияния сложных для классификации объектов, гармонический бустинг нацелен на минимизацию влияния моделей с высокой ошибкой. Это делает метод более устойчивым к выбросам и шуму.

\subsection*{Пример применения}

Рассмотрим задачу классификации на двух классах \( C_1 \) и \( C_2 \). Пусть ансамбль состоит из трёх моделей с вероятностями:
\[
P(C_1 | h_1) = 0.8, \quad P(C_1 | h_2) = 0.6, \quad P(C_1 | h_3) = 0.2.
\]
Итоговое предсказание для класса \( C_1 \) будет:
\[
P(C_1) = \frac{3}{\frac{1}{0.8} + \frac{1}{0.6} + \frac{1}{0.2}} \approx 0.44.
\]

\subsection*{Преимущества и ограничения}

\textbf{Преимущества:}
\begin{itemize}
    \item Устойчивость к шумным данным и выбросам.
    \item Минимизация вклада моделей с высокой ошибкой.
\end{itemize}

\textbf{Ограничения:}
\begin{itemize}
    \item Высокая вычислительная сложность.
    \item Чувствительность к выбору базовых моделей.
\end{itemize}

\section*{Задачи}

\subsection*{Задача 1: Теоретическое доказательство свойства гармонического среднего}

Докажите, что гармоническое среднее минимизирует влияние на общую ошибку моделей с большими значениями ошибки \( e_i \).

\textbf{Решение:}
\begin{enumerate}
    \item Запишем общую ошибку ансамбля:
    \[
    E = \frac{n}{\sum_{i=1}^n \frac{1}{e_i}}.
    \]
    \item Рассмотрим случай, когда одна из ошибок \( e_i \) существенно больше других. Тогда:
    \[
    \sum_{i=1}^n \frac{1}{e_i} \approx \frac{1}{e_1} + \frac{1}{e_2} + \ldots + \frac{1}{e_k} + \frac{1}{e_{\text{max}}},
    \]
    где \( e_{\text{max}} \gg e_k \).
    \item Гармоническое среднее делает вклад \( \frac{1}{e_{\text{max}}} \) минимальным, сохраняя значительное влияние \( \frac{1}{e_k} \), что уменьшает общий эффект высоких ошибок на ансамбль.
\end{enumerate}

\subsection*{Задача 2: Алгоритм на данных}

Даны предсказания трёх моделей на выборке:
\[
P(C_1 | h_1) = [0.9, 0.3, 0.6], \quad P(C_1 | h_2) = [0.8, 0.4, 0.7], \quad P(C_1 | h_3) = [0.7, 0.2, 0.5].
\]
Используя гармонический бустинг, вычислите итоговое предсказание ансамбля.

\textbf{Решение:}
\begin{enumerate}
    \item Рассчитаем гармоническое среднее для каждого объекта:
    \[
    P(C_1 | x_1) = \frac{3}{\frac{1}{0.9} + \frac{1}{0.8} + \frac{1}{0.7}} \approx 0.8,
    \]
    \[
    P(C_1 | x_2) = \frac{3}{\frac{1}{0.3} + \frac{1}{0.4} + \frac{1}{0.2}} \approx 0.26,
    \]
    \[
    P(C_1 | x_3) = \frac{3}{\frac{1}{0.6} + \frac{1}{0.7} + \frac{1}{0.5}} \approx 0.57.
    \]
    \item Итоговое предсказание: \( [0.8, 0.26, 0.57] \).
\end{enumerate}

\subsection*{Задача 3: Сравнение с арифметическим средним}

Сравните результаты гармонического и арифметического средних для трёх моделей с предсказаниями:
\[
P(C_1 | h_1) = 0.9, \quad P(C_1 | h_2) = 0.1, \quad P(C_1 | h_3) = 0.8.
\]

\textbf{Решение:}
\begin{enumerate}
    \item \textbf{Арифметическое среднее:}
    \[
    P(C_1) = \frac{0.9 + 0.1 + 0.8}{3} = 0.6.
    \]
    \item \textbf{Гармоническое среднее:}
    \[
    P(C_1) = \frac{3}{\frac{1}{0.9} + \frac{1}{0.1} + \frac{1}{0.8}} \approx 0.27.
    \]
    \item Гармоническое среднее уменьшает вклад модели \( h_2 \) с высокой ошибкой \( (P(C_1) = 0.1) \), что приводит к более устойчивому предсказанию.
\end{enumerate}
